%%%%%%%%%%%%%%%%%%%%%%%%%%%%%%%%%%%%%%%%%
% Developer CV
% LaTeX Template
% Version 1.0 (28/1/19)
%
% This template originates from:
% http://www.LaTeXTemplates.com
%
% Authors:
% Jan Vorisek (jan@vorisek.me)
% Based on a template by Jan Küster (info@jankuester.com)
% Modified for LaTeX Templates by Vel (vel@LaTeXTemplates.com)
%
% License:
% The MIT License (see included LICENSE file)
%
%%%%%%%%%%%%%%%%%%%%%%%%%%%%%%%%%%%%%%%%%

%----------------------------------------------------------------------------------------
%	PACKAGES AND OTHER DOCUMENT CONFIGURATIONS
%----------------------------------------------------------------------------------------

\documentclass[9pt]{developercv} % Default font size, values from 8-12pt are recommended

%----------------------------------------------------------------------------------------

\begin{document}

%----------------------------------------------------------------------------------------
%	TITLE AND CONTACT INFORMATION
%----------------------------------------------------------------------------------------

\begin{minipage}[t]{0.45\textwidth} % 45% of the page width for name
	\vspace{-\baselineskip} % Required for vertically aligning minipages
	
	% If your name is very short, use just one of the lines below
	% If your name is very long, reduce the font size or make the minipage wider and reduce the others proportionately
	\colorbox{black}{{\Huge\textcolor{white}{\textbf{\MakeUppercase{Humberto de Jesus}}}}} % First name
	
	\colorbox{black}{{\Huge\textcolor{white}{\textbf{\MakeUppercase{Flores Acuña}}}}} % Last name
	
	\vspace{6pt}
	
	{\huge Developer / Sysadmin} % Career or current job title
\end{minipage}
\begin{minipage}[t]{0.275\textwidth} % 27.5% of the page width for the first row of icons
	\vspace{-\baselineskip} % Required for vertically aligning minipages
	
	% The first parameter is the FontAwesome icon name, the second is the box size and the third is the text
	% Other icons can be found by referring to fontawesome.pdf (supplied with the template) and using the word after \fa in the command for the icon you want
	\icon{MapMarker}{12}{Estado de México, México}\\
	\icon{Phone}{12}{+52 1 55 49266159}\\
	\icon{At}{12}{\href{mailto:hdjesus.flores@gmail.com}{hdjesus.flores@gmail.com}}\\	
\end{minipage}
\begin{minipage}[t]{0.275\textwidth} % 27.5% of the page width for the second row of icons
	\vspace{-\baselineskip} % Required for vertically aligning minipages
	
	% The first parameter is the FontAwesome icon name, the second is the box size and the third is the text
	% Other icons can be found by referring to fontawesome.pdf (supplied with the template) and using the word after \fa in the command for the icon you want
	% \icon{Globe}{12}{\href{https://alyx.vance.me}{alyx.vance.me}}\\
	\icon{Github}{12}{\href{https://github.com/stokekld}{github.com/stokekld}}\\
	% \icon{Twitter}{12}{\href{https://twitter.com/@alyxvance}{@alyxvance}}\\
\end{minipage}

\vspace{0.5cm}

%----------------------------------------------------------------------------------------
%	INTRODUCTION, SKILLS AND TECHNOLOGIES
%----------------------------------------------------------------------------------------

% \cvsect{Who Am I?}
\cvsect{Summary}

\begin{minipage}[t]{0.4\textwidth} % 40% of the page width for the introduction text
	\vspace{-\baselineskip} % Required for vertically aligning minipages
	
	% \lorem \lorem \lorem \lorem \lorem\\ % Dummy text
	Passionate Computer Engineer interested in being part of companies or communities with solutions based in Free and Open Source Software, providing my skills as a developer and system administrator to generate excellent results, obtaining experience to be a better professional and human being.   
\end{minipage}
\hfill % Whitespace between
\begin{minipage}[t]{0.5\textwidth} % 50% of the page for the skills bar chart
	\vspace{-\baselineskip} % Required for vertically aligning minipages
	\begin{barchart}{5.5}
		\baritem{C/C++, Python}{90}
%		\baritem{PHP, Javascript}{50}
		\baritem{GNU/Linux}{95}
		\baritem{LXC, Docker}{80}
		\baritem{Git}{85}
		\baritem{Golang}{10}
	\end{barchart}
\end{minipage}

%----------------------------------------------------------------------------------------
%	EXPERIENCE
%----------------------------------------------------------------------------------------

\cvsect{Experience}

\begin{entrylist}
	\entry
%		{July 2019\\-\\Present\\\\\footnotesize{part time}}
		{July 2019\\-\\Present}
		{IoT developer}
		{Sky Solutions UAV}
		{In the last months, I've been working in the development of an autonomous drone that obtains topographic maps of a specific area. For this task, I've been developing a program written in C++ that controls the fly missions and the obtaining of the pictures.\\ \texttt{Raspberry Pi}\slashsep\texttt{Raspbian}\slashsep\texttt{Python}\slashsep\texttt{C++}\slashsep\texttt{ROS/ROS2}}
	\entry
		{June 2018\\-\\July 2019}
		{IoT developer}
		{Bitwise Integrated Technologies}
		{I participated in the development of an IoT device for the agriculture industry, where my main function was to create the communication between the device and the server and the behavior of the device with its hardware elements. I achieved this work creating a minimum GNU/Linux OS. This OS contained services of systemd that I programmed in C language for manage of MQTT protocol.\\ \texttt{Raspberry Pi}\slashsep\texttt{Debootstrap}\slashsep\texttt{C}}
	\entry
		{March 2016\\-\\June 2018}
		{Sysadmin}
		{UCA Consejos Académicos de Área, UNAM}
		{At the beginning of my work, I set up the servers in different areas of the company (network servers, web servers, database servers) to improve the network performance and its monitoring. After that, I was the Systems Administrator of these servers.\\ \texttt{GNU/Linux}\slashsep\texttt{LXC}\slashsep\texttt{Docker}}
	\entry
		{July 2014\\-\\March 2015}
		{C Developer}
		{SOFTCERT}
		{I was in charge of the servers that make the administration of gas stations. My function was to create commands that calculated liquid dispatched and made backups in the database. At the end of the day, they sent the sales numbers to PEMEX servers. I programmed the commands in C language paying attention to the security of the servers and its corresponding regulations.\\ \texttt{GNU/Linux}\slashsep\texttt{C}}
%	\entry
%		{June 2011\\-\\December 2014}
%		{Social Services / Intern}
%		{DICYG, FACULTAD DE INGENIERÍA, UNAM}
%		{I was part of the team of interns of the Computer Unit of the DICYG, carrying out the role of Programmer Analyst, making systems to measure according to the specific needs of the division, in addition to maintaining existing systems. I also held the position of administrator of the servers where these systems are hosted with their respective databases, likewise I continued to participate slightly in the organization and maintenance of the data network of the building.\\ \texttt{Tech Support}\slashsep\texttt{PHP}}
%	\entry
%		{May 2010\\-\\April 2011}
%		{PHP Developer}
%		{ECOVIVIENDA}
%		{The main activity that I performed in the company was to provide support and development to the modules of the system in charge of the management of all areas of the company (sales, legal, comptroller and administration), as well as participating in the creation of modules. I also solved frequent problems in the data network and provided technical support to computer and office equipment.\\ \texttt{PHP}\\\\}
\end{entrylist}

%----------------------------------------------------------------------------------------
%	EDUCATION
%----------------------------------------------------------------------------------------

\cvsect{Education}

\begin{entrylist}
	\entry
		{2016}
		{Linux on Embedded Systems / Diploma}
		{Facultad de Ingeniería, UNAM}
		{}
	\entry
		{2006 - 2012}
		{Computer Engineering / Bachelor's Degree}
		{Facultad de Ingeniería, UNAM}
		{}
\end{entrylist}

%----------------------------------------------------------------------------------------
%	ADDITIONAL INFORMATION
%----------------------------------------------------------------------------------------

\begin{minipage}[t]{0.3\textwidth}
	\vspace{-\baselineskip} % Required for vertically aligning minipages

	\cvsect{Languages}
	
	\textbf{Spanish} - native\\
	\textbf{English} - proficient\\
\end{minipage}

%----------------------------------------------------------------------------------------

\end{document}
